\documentclass[leqno]{article}
\usepackage[utf8]{inputenc}
\usepackage{enumitem}
\usepackage{tikz}
\usepackage[parfill]{parskip} % Don't start new paragraph with tab.
\usepackage{amsmath} % For \tag and \eqref

\title{Computationele logica}
\author{
    \small Kamans, Jim\\
    \texttt{10302905}
    \and
    \small Hendrikse, Mila\\
    \texttt{00000000}
    \and
    \small Roosingh, Sander\\
    \texttt{11983957}
    \and
    \small Schenk, Stefan\\
    \texttt{11881798}
}
\date{November 2017}

\begin{document}

\maketitle


%%%%%%%%%%%%%%%%
%% Exercise 1 %%
%%%%%%%%%%%%%%%%
\section{Exercise 1}

\begin{enumerate}

    \item The sentence $\theta$ encoding all information: \\

    \item A representation of the situation model \textbf{M}: \\
    $\mathcal{A}$ = \{a, b, q\} the agents Alice, Bob, and the Queen \\
    $\Phi$ = $\{r_a, w_a, r_b, w_b\}$ the colors of the hats for a and b \\

    This is an epistemic model: YES / NO \\

    \item Seperately a and b look in their mirrors and see their red hats, represented in the event model $\Sigma$ with four actions: \\

    This is an epistemic model: YES / NO \\
    This is a doxasic model: YES / NO \\

    \item The update product of the two models \textbf{M} $\bigotimes$ $\Sigma$ : \\

    This is an epistemic model: YES / NO \\
    This is a doxasic model: YES / NO \\

\end{enumerate}


%%%%%%%%%%%%%%%%
%% Exercise 2 %%
%%%%%%%%%%%%%%%%
\section{Exercise 2}

\begin{enumerate}

    \item There are ? possible worlds.
    \item
    \item
    \item
    \item

\end{enumerate}


%%%%%%%%%%%%%%%%
%% Exercise 3 %%
%%%%%%%%%%%%%%%%
\section{Exercise 2}

\begin{enumerate}

    \item
    \item
    \item
    \item
    \item
    \item
    \item
    \item
    \item

\end{enumerate}


\end{document}


